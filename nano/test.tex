\documentclass[aspectratio=1610]{beamer}
\usefonttheme{professionalfonts}
\usetheme{metropolis}
\usepackage{polyglossia}
\setmainlanguage{english}
\usepackage{amsmath}
\usepackage{amssymb}
\usepackage{mathtools}
\usepackage{graphicx}
\usepackage[version=4]{mhchem}
\usepackage[
  math-style=ISO,
  bold-style=ISO,
  sans-style=italic,
  nabla=upright,
]{unicode-math}
\setmathfont{Latin Modern Math}
\usepackage{blindtext}
\usepackage{fontspec}
\title{Photonic Crystal}
\subtitle{and how you can use them for virus detection}
\date{\today}
\author{Steven Becker}
\usepackage{siunitx}
\AtBeginDocument{
\sisetup{
math-rm=\mathrm,
math-micro=μ,
}
}
\usepackage{framed}
\usepackage{biblatex}
\addbibresource{lit.bib}
\usepackage{booktabs}

\begin{document}
%
%%Chapters: 5.3.2.3, 7.3
%%https://www.ncbi.nlm.nih.gov/pmc/articles/PMC3937800/
%\frame{\maketitle}
%
%
%\begin{frame}{Overview}
%\begin{itemize}
%  \setlength\itemsep{1.2em}
%  \item{Virus detection in generell}
%  \item{Photonic crystal}
%  \item{Biosensing with 2D-PhCs}
%\end{itemize}
%
%\end{frame}
%
%\begin{frame}{Virus}
%  \begin{columns}
%
%  \begin{column}{0.49\textwidth}
%    \begin{itemize}
%      \setlength\itemsep{1.2em}
%      \item{no own metabolism}
%      \item{need a host to survive} % no own metabolism
%      \item{size of $\num{20}\,-\, \num{350}\,\si{\nano\meter}$}
%    \end{itemize}
%  \end{column}
%
%  \begin{column}{0.49\textwidth}
%    \begin{figure}
%      \centering
%      \includegraphics[width=0.8\textwidth]{./bilder/virus.png}
%      \caption{Electron microscop picture of a \emph{bluetongue virus}. The bar is $\SI{50}{\nano\meter}$ long \cite{virus}.}
%      \label{fig: 2d_photonic_crystal}
%    \end{figure}
%  \end{column}
%
%  \end{columns}
%\end{frame}
%
%\begin{frame}{Virus detection in generell}
%  \begin{itemize}
%    \setlength\itemsep{1.2em}
%    \item{\emph{Electron microscopy} - observe viruses in a sample   }
%    \item{\emph{Gene sequencing} - out of a sample with millions of bases, you can detect the pathogen's bases   }
%    \item{\emph{ Antibody detection} - immune system produce antibodies to fight the virus }
%    \end{itemize}
%\end{frame}
%
%\begin{frame}{Photonic crystal}
%  \begin{columns}
%
%    \begin{column}{0.49\textwidth}
%    \begin{itemize}
%      \setlength\itemsep{1.2em}
%      \item{periodic dielectric microstructure \, ( 1D, 2D and 3D)}
%      \item{propagation of the electromagnetic wave depends on the wavelength}
%    \end{itemize}
%    \end{column}
%
%    \begin{column}{0.49\textwidth}
%    \begin{figure}
%      \centering
%      \includegraphics[width=1\textwidth]{./bilder/photonic_crystal_model.png}
%      \caption{Example of 1D (a), 2D (b) and 3D (c) photonic chrystals. \cite{intro_pho}.}
%      \label{fig: photonic_crystal}
%    \end{figure}
%  \end{column}
%
%  \end{columns}
%
%\end{frame}
%
%\begin{frame}{Photonic crystal}
%
%  \begin{columns}
%
%    \begin{column}{0.49\textwidth}
%      \begin{figure}
%        \centering
%        \includegraphics[width=1\textwidth]{./bilder/band_structure.png}
%        \caption{Band structure of an 1D photonic crystal. \cite{intro_pho}.}
%        \label{fig: band_structure}
%      \end{figure}
%    \end{column}
%
%    \begin{column}{0.49\textwidth}
%    \begin{itemize}
%      \setlength\itemsep{1.2em}
%      \item{characterized by their band structure ( c. fig. \ref{fig: band_structure}) }
%    \end{itemize}
%    \begin{equation*}
%      \lambda=\frac{2\pi c}{\omega}\qquad \omega =2\pi f
%    \end{equation*}
%    \end{column}
%
%  \end{columns}
%
%\end{frame}
%
%\begin{frame}{Biosensing with 2D-PhCs}
%
%  \begin{columns}
%
%    \begin{column}{0.49\textwidth}
%    \begin{itemize}
%      \setlength\itemsep{1.2em}
%      \item{\emph{only structure} - no wavelength is transmitted }
%      \item{\emph{structure + point defect} - only one wavelength is transmitted (result of the point defect)}
%      \item{\emph{structure + point + waveguide} - only one wavelength is absorbed (c. fig. \ref{fig: 2d_photonic_crystal} )}
%    \end{itemize}
%    \end{column}
%
%    \begin{column}{0.49\textwidth}
%    \begin{figure}
%      \centering
%      \includegraphics[width=0.8\textwidth]{./bilder/2dphc_waveguide_point_defect.png}
%      \caption{2D photonic crystal with a punctually defect and a waveguide. \cite{nano}.}
%      \label{fig: 2d_photonic_crystal}
%    \end{figure}
%  \end{column}
%
%  \end{columns}
%
%\end{frame}
%
%
%
%\begin{frame}{Biosensing with 2D-PhCs}
%  \begin{columns}
%
%  \begin{column}{0.49\textwidth}
%    \begin{itemize}
%      \setlength\itemsep{1.2em}
%      \item{strong electric field nearby the point defect inceases the binding sensetivity (c. fig. \ref{fig: e_field})}
%      \item{binding biomaterials changes the local refractive index, causes a \emph{red-shift}}
%    \end{itemize}
%  \end{column}
%
%  \begin{column}{0.49\textwidth}
%    \begin{figure}
%      \centering
%      \includegraphics[width=0.5\textwidth]{./bilder/multiplexing_photon.jpg}
%      \caption{(a) SEM image of a fabricated defect-coupled PhC sensor. (b) 2D simulation results for the |E|2 field profile of the PhC structure for the lowest frequency cavity resonance.
%                \cite{test_Structure}.}
%      \label{fig: e_field}
%    \end{figure}
%  \end{column}
%
%\end{columns}
%\end{frame}
%
%

%\begin{frame}{Biosensing with 2D-PhCs}
%  \begin{columns}
%
%  \begin{column}{0.49\textwidth}
%    \begin{itemize}
%    \setlength\itemsep{1.2em}
%    \item{ surface modification gives the option to detect a specific biological particle}
%    \item{ \emph{modification} - first oxidation and silanization, second specific biotin}
%    \item{ multiplexing increases the spectrum of detection}
%  \end{itemize}
%  \end{column}
%
%  \begin{column}{0.49\textwidth}
%  \begin{figure}
%    \centering
%    \includegraphics[width=1\textwidth]{./bilder/biotin.png}
%    \caption{Biotin increases the sensetivity of detection \cite{nano}.}
%    \label{fig: redshift}
%  \end{figure}
%  \end{column}
%
%  \end{columns}
%\end{frame}
%
%\begin{frame}{Biosensing with 2D-PhCs}
%\begin{columns}
%  \begin{column}{0.49\textwidth}
%    \begin{figure}
%      \centering
%      \includegraphics[width=0.5\textwidth]{./bilder/reflektion.png}
%      \caption{Surface modification of a PhC allow a virurs detection. \cite{nano}.}
%      \label{fig: virus detection}
%    \end{figure}
%  \end{column}
%
%  \begin{column}{0.49\textwidth}
%    \begin{figure}
%      \centering
%      \includegraphics[width=0.5\textwidth]{./bilder/virus_surface.jpg}
%      \caption{(a) Scanning Electron Microscopy (SEM) imaging of photonic crystal surfaces with captured HIV-1 (b) Close-up image of a single virus attached to the PC surface. \cite{virus_surface}.}
%      \label{fig: virus_on_surface}
%    \end{figure}
%  \end{column}
%\end{columns}
%
%\end{frame}

\begin{frame}{SumUp}
  \begin{itemize}
    \setlength\itemsep{1.2em}
    \item{Photonic crystals have band gaps}
    \item{waveguides and point defects affect the behavior of PhC}
    \item{bionsensing works by measuring the redshift of the sample }
    \item{biotin can increase the sensitivity}
    %\item{PhCs can be used as a rapid, sensitive, inexpensive and portable diagnostic tool in future}
  \end{itemize}
\end{frame}

%\begin{frame}
%  \nocite{*}
%  \printbibliography
%\end{frame}
\end{document}
