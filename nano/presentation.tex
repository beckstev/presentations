\documentclass[aspectratio=1610]{beamer}
\usefonttheme{professionalfonts}
\usetheme{metropolis}
\usepackage{polyglossia}
\setmainlanguage{english}
\usepackage{amsmath}
\usepackage{amssymb}
\usepackage{mathtools}
\usepackage{graphicx}
\usepackage[version=4]{mhchem}
\usepackage[
  math-style=ISO,
  bold-style=ISO,
  sans-style=italic,
  nabla=upright,
]{unicode-math}
\setmathfont{Latin Modern Math}
\usepackage{blindtext}
\usepackage{fontspec}
\title{Photonic Crystal}
\subtitle{and how you can use them for virus detection}
\date{\today}
\author{Steven Becker}
\usepackage{siunitx}
\AtBeginDocument{
\sisetup{
math-rm=\mathrm,
math-micro=μ,
}
}
\usepackage{framed}
\usepackage{biblatex}
\addbibresource{lit.bib}
\usepackage{booktabs}

\begin{document}

%Chapters: 5.3.2.3, 7.3
%https://www.ncbi.nlm.nih.gov/pmc/articles/PMC3937800/
\frame{\maketitle}

\begin{frame}
  Bob is ill.
\end{frame}

\begin{frame}{Overview}
\begin{itemize}
  \setlength\itemsep{1.2em}
  \item{Virus detection in generell}
  \item{Photonic crystal}
  \item{2D-PhCs}
  \item{virus detection in generell}
  \item{How does the detection work?}
\end{itemize}
\end{frame}

\begin{frame}{Virus detection in generell}
  \begin{itemize}
    \setlength\itemsep{1.2em}
    \item{\emph{Electron microscopy} - observe viruses in a sample   }
    \item{\emph{Gene sequencing} - out of a sample with millions of bases, you can detect the pathogen's bases   }
    \item{\emph{ Antibody detection} - immune system produce antibodies to fight the virus }
    \end{itemize}
\end{frame}

\begin{frame}{Photonic crystal}
  \begin{columns}

    \begin{column}{0.49\textwidth}
    \begin{itemize}
      \setlength\itemsep{1.2em}
      \item{periodic dielectric mikrostructure \, ( 1D, 2D and 3D)}
      \item{propagation of the electromagnetiv wave depends on the wavelength}
    \end{itemize}
    \end{column}

    \begin{column}{0.49\textwidth}
    \begin{figure}
      \centering
      \includegraphics[width=1\textwidth]{./bilder/photonic_crystal_model.png}
      \caption{Example of 1D (a), 2D (b) and 3D (c) photonic chrystals. \cite{intro_pho}.}
      \label{fig: photonic_crystal}
    \end{figure}
  \end{column}

  \end{columns}

\end{frame}

\begin{frame}{Photonic crystal}

  \begin{columns}

    \begin{column}{0.49\textwidth}
      \begin{figure}
        \centering
        \includegraphics[width=1\textwidth]{./bilder/band_structure.png}
        \caption{Band structure of an 1D photonic crystial. \cite{intro_pho}.}
        \label{fig: band_structure}
      \end{figure}
    \end{column}

    \begin{column}{0.49\textwidth}
    \begin{itemize}
      \setlength\itemsep{1.2em}
      \item{ characterized by their band structure ( c. \ref{fig: band_structure}) }
    \end{itemize}
    \begin{equation*}
      \lambda=\frac{2\pi c}{\omega}\qquad \omega =2\pi f
    \end{equation*}
    \end{column}

  \end{columns}

\end{frame}

\begin{frame}{2D-PhCs}

\end{frame}

\begin{frame}{How does the detection work?}

\end{frame}

\begin{frame}{Potential}
  \begin{itemize}
    \item { mobile diagnostic tool}
  \end{itemize}
\end{frame}

\begin{frame}
  \nocite{*}
  \printbibliography
\end{frame}
\end{document}
