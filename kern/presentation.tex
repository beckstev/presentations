\documentclass[aspectratio=1610]{beamer}
\usefonttheme{professionalfonts}
\usetheme{metropolis}
\usepackage{polyglossia}
\setmainlanguage{english}
\usepackage{amsmath}
\usepackage{amssymb}
\usepackage{mathtools}
\usepackage{graphicx}
\usepackage[version=4]{mhchem}
\usepackage[
  math-style=ISO,
  bold-style=ISO,
  sans-style=italic,
  nabla=upright,
]{unicode-math}
\setmathfont{Latin Modern Math}
\usepackage{blindtext}
\usepackage{fontspec}
\title{Photonic Crystal}
\subtitle{and how you can use them for virus detection}
\date{\today}
\author{Steven Becker}
\usepackage{siunitx}
\AtBeginDocument{
\sisetup{
math-rm=\mathrm,
math-micro=μ,
}
}
\usepackage{framed}
\usepackage{biblatex}
\addbibresource{lit.bib}
\usepackage{booktabs}

\begin{document}

\frame{\maketitle}



\begin{frame}{ Überblick }
  \begin{itemize}
    \setlength\itemsep{1.2em}
      \item{ Situation in Deutschland }
      \item{ Weg zur Stilllegung }
      \item{ Der Rückbau }
      \item{ Kosten }
  \end{itemize}
\end{frame}



\section{Situation in Deutschland}



\begin{frame}{ Situation in Deutschland }
  \begin{figure}
     \centering
     \includegraphics[width=0.4\textwidth]{./bilder/akw_abschaltung_karte.png}
     \caption{ Auflistung der Abschaltungsjahre von deutschen AKWs \cite{ karte_abschaltungen }. }
     \label{ fig: karte_abschaltungen }
   \end{figure}
\end{frame}



\section{Weg zur Stilllegung}



\begin{frame}{ Weg zur Stilllegung }
  \begin{itemize}
    \setlength\itemsep{1.2em}
      \item{ Stillegungen müssen beantragt werden}
      \item{ Länder sind dafür zuständig}
      \item{ Unterliegt dem Atomrecht}
  \end{itemize}
\end{frame}



\begin{frame}{ Nachbetriebsphase }
  \begin{columns}

    \begin{column}{0.48\textwidth}

        \begin{itemize}
          \setlength\itemsep{1.2em}
          \item{ Abschaltung des Kernreaktors }
          \item{ Dauer von etwa 5 Jahren nach der Abschlatung}
          \item{ Brennelemente müssen noch weiter gekühlt werden }
          \item{ radioaktive Betriebsabfälle werden entfernt }
        \end{itemize}

    \end{column}

    \begin{column}{0.48\textwidth}
      Senkung der durchschnitllichen Aktivität
      \begin{equation*}
        \SI{10e20}{\becquerel} \quad \rightarrow  \quad \SI{10e16}{\becquerel}
      \end{equation*}
    \end{column}

  \end{columns}
\end{frame}



\begin{frame}{Stillegungstrategien - Direkter Abbau}
  \begin{itemize}
    \setlength\itemsep{1.2em}
      \item{ Rückkbau unmittelbar nach Abschaltung }
      \item{ dauert mindestens $10$ Jahre }
      \item{ wird in Deutschland am häufigstens verwendet}
  \end{itemize}
\end{frame}



\begin{frame}{Stillegungstrategien - Sicherer Einschluss}
  \begin{itemize}
    \setlength\itemsep{1.2em}
    \item{ Nach der Abschaltung wird der Reaktor in eine wartungsarmen Zustand gebracht}
    \item{ Dauer von etwa $30$ Jahren}
  \end{itemize}
\end{frame}



\begin{frame}{ Direkter Abbau - Sicherer Einschlus - Ein Vergleich}
  \begin{figure}
     \centering
     \includegraphics[width=0.9\textwidth]{./bilder/vor_nachteile_direkter_einschluss.PNG}
     \caption{ Vor- und Nachteile von Direkter Abbau und Sicheren Einschluss \cite{stilllegung_grs}. }
     \label{ fig: karte_abschaltungen }
   \end{figure}
\end{frame}



\begin{frame}{ Weg zur Stillegung - Direkter Abbau }
  \begin{figure}
     \centering
     \includegraphics[width=0.6\textwidth]{./bilder/stillegungs_zeit_2_kernfragen.PNG}
     \caption{ Zeitlicher Verlauf eines direkten Abbau\cite{stilllegung_grs}. }
     \label{ fig: stillegung }
   \end{figure}
\end{frame}



\begin{frame}{Phase 1}
  \begin{columns}

    \begin{column}{0.48\textwidth}
      \begin{itemize}
        \setlength\itemsep{1.2em}
        \item{ Ausbau von nicht mehr benötigten Teilen z.\, B. Regelstabführungen }
        \item{ Platz schaffen }
      \end{itemize}
    \end{column}

    \begin{column}{0.48\textwidth}





\begin{frame}[allowframebreaks]
  \nocite{*}
  \printbibliography
\end{frame}

\end{document}
